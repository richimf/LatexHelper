\documentclass[12pt]{article}
\usepackage{amsfonts}
\usepackage[top=1in, bottom=1in, left=1in, right=1in]{geometry}
\usepackage{graphicx}

% Definimos macros o ecuaciones que usaremos a lo largo del documento
\def\eq1{y=\frac{x}{x^2+3x+2}}

\begin{document}

% Indice, automaticamente pone los sections y subsections
\tableofcontents

\title{MiTitulo}
\author{Richie}
\date{\today}
\maketitle

\section{Seccion uno (20 pts)}
\subsection{Subseccion}
\subsection{Subseccion}
\section{Seccion dos}
\subsection{Subseccion}


Numeros naturales $\mathbb{N}$, gracias al paquete "amsfonts".
$\mathbb{Z}$

My text \textit{italiano} texto.
Ahora es \textbf{negro}.
Y ahora es  \textsc{MAYUSCULAS}.
Y ahora escrito por un  \texttt{typewriter} \\
Soy un sitio web \texttt{http://www.google.com.mx}\\

Usando Macros, una ecuación definida arriba: $\eq1$
\\
letras grandes \begin{Large}
holaaaaa
\end{Large}

texto \begin{center}
centrado
\end{center}

Texto justificado 
\begin{flushleft}
texto texto texto
\end{flushleft}

Hola que hace $(x+1)$ and $(x+3)$\\

supperscripts:
$$ 2x^2$$
$$ 2x^{3x+4}$$

subscripts:
$$x_1$$
$$x_{12}$$
$${{x_1}_2}_3$$

trig functions:
$$y=\sin{x}$$

log functions:
$$\log{x}$$
$$\ln{x}$$

square roots:
$$\sqrt[3]{x^{3x+2}}$$

fractions:\\
About $\frac{2}{3}$ of the glass if full.\\
\\
Fracciones mas grandes $\displaystyle{\frac{2}{3}}$ of the glass if full.

$$\frac{x}{x^2+x+1}$$

Parentesis:
$$\{a,b,c,d,e\}$$

$$3\left(\frac{2}{3}\right)$$

$$3\left[\frac{2}{3}\right]$$

$$3\left\{\frac{2}{3}\right\}$$

$$\left. \frac{dy}{dx} \right|_{x=1}$$

Tables, c una columna, cc dos columnas...\\
con ampersan cambiamos de columna.\\

\begin{tabular}{|c|ccccc|}
\hline
$x$  & 1 & 2 & 3 & 4 & 5 \\ \hline
$f(x)$  & 11 & 22 & 33 & 43 & 54 \\ \hline
\end{tabular}


Ecuaciones, con amperson se alinean los = :
El asterisco oculta los indices de la derecha
\begin{eqnarray*}
5x^2-9 &=& x + 3 \\
3x^2-9 &=& 12 \\
5x^2-9 &=& x + 3\\
x&\approx&\pm1.2323
\end{eqnarray*}


Enumeraciones:
\begin{enumerate}
\item Pencil
\item calculator
\item ruler
	\begin{enumerate}
	\item adasdads
	\begin{enumerate}
	\item adasdads
	\end{enumerate}
	\item adsdsdsd
	\item adsdsdsd
	\end{enumerate}
\item graph
\end{enumerate}

Usando imagenes
\\
\includegraphics[width=5in]{imagen.png}

\end{document}